\documentclass[12pt]{article}
\usepackage[italian]{babel}
\usepackage{natbib}
\usepackage{url}
\usepackage[utf8x]{inputenc}
\usepackage{amsmath}
\usepackage{graphicx}
\graphicspath{{images/}}
\usepackage{parskip}
\usepackage{fancyhdr}
\usepackage{vmargin}
\usepackage{float}
\usepackage{hyperref}
\setmarginsrb{3 cm}{2.5 cm}{3 cm}{2.5 cm}{1 cm}{1.5 cm}{1 cm}{1.5 cm}

\title{HAR Bayesian Network}								% Title
\author{Artifoni Mattia \\ Brena Luca \\ Bottoni Federico}								% Author
\date{Giugno 2019}											% Date

\makeatletter
\let\thetitle\@title
\let\theauthor\@author
\let\thedate\@date
\makeatother

\pagestyle{fancy}
\fancyhf{}
\rhead{Artifoni M. Brena L. Bottoni F.}
\lhead{\thetitle}
\cfoot{\thepage}

\begin{document}

%%%%%%%%%%%%%%%%%%%%%%%%%%%%%%%%%%%%%%%%%%%%%%%%%%%%%%%%%%%%%%%%%%%%%%%%%%%%%%%%%%%%%%%%%

\begin{titlepage}
	\centering
    \vspace*{0.5 cm}
    \includegraphics[scale = 0.75]{images/LogoBicocca.pdf}\\[1.0 cm]	% University Logo
    \textsc{\LARGE Università degli studi di}\\[0.2 cm]
    \textsc{\LARGE Milano-Bicocca}\\[2.0 cm]	% University Name
	\textsc{\Large F1801Q145}\\[0.5 cm]				% Course Code
	\textsc{\large Modelli probabilistici per le decisioni}\\[0.5 cm]				% Course Name
	\rule{\linewidth}{0.2 mm} \\[0.4 cm]
	{ \huge \bfseries \thetitle}\\
	\rule{\linewidth}{0.2 mm} \\[1.5 cm]

	\begin{minipage}{0.4\textwidth}
		\begin{flushleft} \large
			\emph{Studenti:}\\
			\theauthor
			\end{flushleft}
			\end{minipage}~
			\begin{minipage}{0.4\textwidth}
			\begin{flushright} \large
			\emph{Matricole:} \\
			807466 \\ 808216 \\ 806944
		\end{flushright}
	\end{minipage}\\[2 cm]

	{\large \thedate}\\[2 cm]

	\vfill

\end{titlepage}

%%%%%%%%%%%%%%%%%%%%%%%%%%%%%%%%%%%%%%%%%%%%%%%%%%%%%%%%%%%%%%%%%%%%%%%%%%%%%%%%%%%%%%%%%

\tableofcontents
\pagebreak

%%%%%%%%%%%%%%%%%%%%%%%%%%%%%%%%%%%%%%%%%%%%%%%%%%%%%%%%%%%%%%%%%%%%%%%%%%%%%%%%%%%%%%%%%



\section{Abstract}

During the last 5 years, research on Human Activity Recognition (HAR)\cite{site:HAR} has reported on systems showing good overall recognition performance. As a consequence, HAR has been considered as a potential technology for e-health systems. Here, we propose a machine learning based HAR classifier\cite{Paper}. We also provide a full experimental description that contains the HAR wearable devices setup and a public domain dataset comprising 165,633 samples. We consider 5 activity classes, gathered from 4 subjects wearing accelerometers mounted on their waist, left thigh, right arm, and right ankle. As basic input features to our classifier we use 12 attributes derived from a time window of 150ms. Finally, the classifier uses a committee AdaBoost that combines ten Decision Trees. The observed classifier accuracy is 99.4%.


\section{Introduzione}



\subsection{Dominio di riferimento}


\subsection{Obiettivi dell'elaborato}



\subsection{Ipotesi e assunzioni}






\clearpage
\section{Conclusioni}

AAAA

\newpage
\bibliographystyle{plain}
\bibliography{biblist}

\end{document}
