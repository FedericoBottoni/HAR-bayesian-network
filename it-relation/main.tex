\documentclass[12pt]{article}
\usepackage[italian]{babel}
\usepackage{natbib}
\usepackage{url}
\usepackage[utf8x]{inputenc}
\usepackage{amsmath}
\usepackage{graphicx}
\graphicspath{{images/}}
\usepackage{parskip}
\usepackage{fancyhdr}
\usepackage{vmargin}
\usepackage{float}
\usepackage{hyperref}
\setmarginsrb{3 cm}{2.5 cm}{3 cm}{2.5 cm}{1 cm}{1.5 cm}{1 cm}{1.5 cm}

\title{HAR Bayesian Network}								% Title
\author{Artifoni Mattia \\ Brena Luca \\ Bottoni Federico}								% Author
\date{Giugno 2019}											% Date

\makeatletter
\let\thetitle\@title
\let\theauthor\@author
\let\thedate\@date
\makeatother

\pagestyle{fancy}
\fancyhf{}
\rhead{Artifoni M. Brena L. Bottoni F.}
\lhead{\thetitle}
\cfoot{\thepage}

\begin{document}

%%%%%%%%%%%%%%%%%%%%%%%%%%%%%%%%%%%%%%%%%%%%%%%%%%%%%%%%%%%%%%%%%%%%%%%%%%%%%%%%%%%%%%%%%

\begin{titlepage}
	\centering
    \vspace*{0.5 cm}
    \includegraphics[scale = 0.75]{images/LogoBicocca.pdf}\\[1.0 cm]	% University Logo
    \textsc{\LARGE Università degli studi di}\\[0.2 cm]
    \textsc{\LARGE Milano-Bicocca}\\[2.0 cm]	% University Name
	\textsc{\Large F1801Q145}\\[0.5 cm]				% Course Code
	\textsc{\large Modelli probabilistici per le decisioni}\\[0.5 cm]				% Course Name
	\rule{\linewidth}{0.2 mm} \\[0.4 cm]
	{ \huge \bfseries \thetitle}\\
	\rule{\linewidth}{0.2 mm} \\[1.5 cm]

	\begin{minipage}{0.4\textwidth}
		\begin{flushleft} \large
			\emph{Studenti:}\\
			\theauthor
			\end{flushleft}
			\end{minipage}~
			\begin{minipage}{0.4\textwidth}
			\begin{flushright} \large
			\emph{Matricole:} \\
			807466 \\ 808216 \\ 806944
		\end{flushright}
	\end{minipage}\\[2 cm]

	{\large \thedate}\\[2 cm]

	\vfill

\end{titlepage}

%%%%%%%%%%%%%%%%%%%%%%%%%%%%%%%%%%%%%%%%%%%%%%%%%%%%%%%%%%%%%%%%%%%%%%%%%%%%%%%%%%%%%%%%%

\tableofcontents
\pagebreak

%%%%%%%%%%%%%%%%%%%%%%%%%%%%%%%%%%%%%%%%%%%%%%%%%%%%%%%%%%%%%%%%%%%%%%%%%%%%%%%%%%%%%%%%%



\section{Introduzione}
Il progetto ha l'obiettivo di creare un modello di Rete Bayesiana capace di predirre il tipo di azione che sta effettuando un ipotetico individuo che indossa il "HAR wearable devices setup", una particolare sistema indossabile composto da 4 accelerometri che permette di analizzare i vettori accelerazione dei sensori in questione. Viene fornito dal progetto di riferimento\cite{HAR} un dataset contenente dati sufficienti per effettuare training e testing del modello

\subsection{Dominio di riferimento}
La natura dei dati utilizzati è definita nel paper\cite{Paper} del progetto di provenienza. La singola entry del dataset rappresenta uno snapshot acquisito dai sensori e consiste in:
\begin{itemize}
	\item user: stringa rappresentante l'username dell'individuo in oggetto
	\item geneder: stringa relativa al suo genere
	\item age: intero dell'età
	\item how\_tall\_in\_meters: decimale che esprime l'altezza
	\item weight: intero riguardo al peso
	\item body\_mass\_index: decimale rappresentante la distribuzione del peso
	\item xi: intero che esprime la componente x del vettore accelerazione nel sensore i-esimo
	\item yi: intero che esprime la componente y del vettore accelerazione nel sensore i-esimo
	\item zi: intero che esprime la componente z del vettore accelerazione nel sensore i-esimo
	\item class: stringa, il target del modello sviluppato: rappresenta l'azione/situazione dell'individuo al momento dello snapshot, può assumere il valore di "walking", "standing", "standingup", "sitting" e "sittingdown"
\end{itemize}


\subsection{Ipotesi e assunzioni}
Durante lo studio del caso sono state discrimate le features utili al training della rete (i vettori dei sensori) da quelle assunte come superflue (user, gender, age, weight, body\_mass\_index) le quali potrebbero essere utilizzate per specializzarla ulteriormente.\newline
La scelta riguardo all'attributo \emph{how\_tall\_in\_meters} non è stata particolarmente immediata dato che il training set considera un range di 13cm (1.58m - 1.71m) che distribuiti in un corpo umano non crea l'informazione necessaria per poter affermare che tutti i sensori si trovano 13cm più o meno vicini al terreno. La rete dovrebbe essere comunque in grado di predirre le azioni di un bambino, il quale ha altezza decisamente inferiore rispetto a quella precedentemente descritta, perciò [...] TODO [...] 

\section{Scelte di design}
\subsection{Analisi statistica e qualitativa}
\begin{figure}[H]
	\centering
	{\includegraphics[width=1\textwidth]{images/dataset.JPG}}
	\caption{Dataset dopo lo shuffle}
	\label{fig:dataset}
\end{figure}
Il dataset si presenta come in figura \ref{fig:dataset} dopo una prima fase di pulizia, in cui sono stati individuati alcuni caratteri non necessari tra i campi e timestamp inaspettati tra le entry della tabella, ed una seconda di shuffle, nella quale i record sono stati randomizzati.\newline
E' stata effettuata inoltre una fase di analisi statistico-descrittiva considerando le features che assumono valori in range indefiniti per cercare di individuare qualche distribuzione particolare o comportamento anomalo.


\begin{table}[h]
	\caption{Analisi descrittiva del dataset}\label{tab:analytics}
	\begin{tabular}{|l|l|l|l|l|l|}
		\hline
		Campo & Min & Max & Media & Moda & DevStd \\
		\hline
		x1 & -306 & 509 & -6.649327127 & -1 & 11.61623803 \\
		y1 & -271 & 533 & 88.29366732 & 95 & 23.89582898 \\
		z1 & -603 & 411 & -93.16461092 & -98 & 39.40942342 \\
		x2 & -494 & 473 & -87.82750418 & -492 & 169.4351938 \\
		y2 & -517 & 295 & -52.06504742 & -516 & 205.1597632 \\
		z2 & -617 & 122 & -175.0552004 & -616 & 192.8166147 \\
		x3 & -499 & 507 & 17.42351464 & 38 & 52.63538753 \\
		y3 & -506 & 517 & 104.5171675 & 108 & 54.15584251 \\
		z3 & -613 & 410 & -93.88172647 & -102 & 45.38964613 \\
		x4 & -702 & -13 & -167.6414483 & -164 & 38.31134199 \\
		y4 & -526 & 86 & -92.62517131 & -94 & 19.96861022 \\
		z4 & -537 & -43 & 88.29366732 & -162 & 13.22102006 \\
		
		\hline
	\end{tabular}
\end{table}

Dalla tabella \ref{tab:analytics} si può notare che i range di variabilità degli attributi non seguono comportamenti particolari, tanto meno le distribuzioni che in alcuni casi sono caratterizzati da deviazione standard particolarmente bassa (come il caso di \emph{x1}) mentre in altri casi molto alta (come \emph{y2}).

\subsection{Normalizzazione}
\subsection{Discretizzazione}

\section{I modelli di rete}
\subsection{pgmpy}
Il software scelto è pgmpy\cite{pgmpy} di Python, una libreria che permette di modellare le dipendenze in modo agile, stimare le CPT delle variabili sfruttando dei metodi che accettano il dataset ed effettuare inferenze dichiarando la variabile di query e le evidenze. \linebreak
Utilizzando la libreria ci siamo resi conto di come sia performante utilizzando modelli semplici e correlati da pochi record, tuttavia appena è avvenuta l'esecuzione della stima delle CPT nel primo modello completo ideato abbiamo riscontrato le prime difficoltà: il modello decisamente complesso viene bloccato da Python date le eccessive combinazioni possibli. Abbiamo quindi tentato di semplificare la rete.

\subsection{Il modello correlato}
\begin{figure}[H]
	\centering
	{\includegraphics[width=1\textwidth]{images/corr05.JPG}}
	\caption{Indice di correlazione di Paerson calcolato su tuttle le combinazioni di componenti}
	\label{fig:corr}
\end{figure}



\clearpage
\section{Risultati e conclusioni}

\newpage
\bibliographystyle{plain}
\bibliography{biblist}

\end{document}
